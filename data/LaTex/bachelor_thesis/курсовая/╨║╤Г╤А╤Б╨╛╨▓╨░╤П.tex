% Тут используется класс, установленный на сервере Papeeria. На случай, если
% текст понадобится редактировать где-то в другом месте, рядом лежит файл matmex-diploma-custom.cls
% который в момент своего создания был идентичен классу, установленному на сервере.
% Для того, чтобы им воспользоваться, замените matmex-diploma на matmex-diploma-custom
% Если вы работаете исключительно в Papeeria то мы настоятельно рекомендуем пользоваться
% классом matmex-diploma, поскольку он будет автоматически обновляться по мере внесения корректив
%

% По умолчанию используется шрифт 14 размера. Если нужен 12-й шрифт, уберите опцию [14pt]
\documentclass[14pt]{matmex-diploma}
%\documentclass{matmex-diploma-custom}

\usepackage{float}


\begin{document}
% Год, город, название университета и факультета предопределены,
% но можно и поменять.
% Если англоязычная титульная страница не нужна, то ее можно просто удалить.
\filltitle{ru}{
    chair              = {Кафедра Информационно-Аналитических Систем},
    title              = {Рекомендательная система для образовательного контента},
    % Здесь указывается тип работы. Возможные значения:
    %   coursework - Курсовая работа
    %   diploma - Диплом специалиста
    %   master - Диплом магистра
    %   bachelor - Диплом бакалавра
    type               = {coursework},
    position           = {студента},
    group              = 441,
    author             = {Волжина Елена Григорьевна},
    supervisorPosition = {ст. преп.},
    supervisor         = {Ярыгина А.\,С.},
%    reviewerPosition   = {ст. преп.},
%    reviewer           = {Вяххи Н.\,И.},
%    chairHeadPosition  = {профессор},
%    chairHead          = {Новиков Б.\,А.},
    university         = {Санкт-Петербургский Государственный Университет},
    faculty            = {Математико-механический факультет},
    city               = {Санкт-Петербург},
    year               = {2015}
}
\maketitle
\tableofcontents


\section*{Введение}
В современном мире онлайн-образование постепенно становится все более популярным. Возможность учиться у профессоров ведущих учебных заведений, изучать новые области, получать нужные в работе знания, не выходя из дома, привлекает большое количество людей. 
\\\indent Онлайн-обучение обычно проходит в форме массовых открытых онлайн-курсов (MOOC, massive open online course). Обычно это подразумевает видео, слайды и текстовый контент, подготовленные преподавателем, а также задачи для проверки знаний, которые обычно проверяются автоматически, но также возможна проверка студентами работ своих товарищей. В качестве задач могут быть предложены самые разнообразные типы заданий: от простого выбора правильного ответа до задач на программирование и написания эссе.
\\\indent У онлайн-образования есть свои особенности, отличающие его от обычного, оффлайн-образования. Среди плюсов, во-первых, уже упомянутая выше доступность каждому, у кого есть доступ к интернету. Во-вторых, это почти неограниченная масштабируемость: благодаря автоматизированной проверке задач на курсе могут одновременно учиться тысячи человек, что несопоставимо с обычными курсами в учебных аудиториях. В-третьих, каждый студент может выбирать удобное для себя время и темп для прохождения материала.
\\\indent В то же время в онлайн-обучении есть и минусы. В отличие от традиционного образования, где у студента всегда есть мотивация в виде оценки его академической успеваемости, в случае онлайн-курсов нет никаких штрафов за не пройденный курс. Из-за этого доля закончивших курс из записавшихся на него редко превышает $10\%$. Помимо этого, из-за большого числа учащихся у преподавателя нет никакой возможности уделять индивидуальное внимание каждому студенту сообразно его уровню и возможностям.
\\\indent Наиболее известные на данный момент платформы с онлайн-курсами и их посещаемость представлены в таблице ниже.

\begin{table}[H]
\caption{Платформы с онлайн-курсами}
\label{tabular:statistic}
\begin{center}
\begin{tabular}{c|c|c}
\hline
Название & Год запуска & Зарегистрировано пользователей (2015) \\
\hline
Coursera\cite{coursera} & 2012 & 15 млн \\
edX\cite{edx} & 2012 & 5 млн \\
Udacity\cite{udacity} & 2012 & 1.6 млн \\
Stepic.org\cite{stepic} & 2013 & 130 тыс. \\

\end{tabular}
\end{center}
\end{table}

\\\indent Учитывая разнообразие тем онлайн-курсов, отсутствие структурированной программы и различные уровни сложности курсов, очень важно уметь направить пользователя именно к тому материалу, который будет ему интересен и понятен. Для этого естественно применять рекомендательную систему.
\\\indent Рекомендательные системы прочно вошли в нашу жизнь. Особенно они актуальны в онлайн-магазинах, на новостных сайтах, а также на сайтах с развлекательным контентом: книгами, фильмами и музыкой. Одним из ярких примеров является сайт Netflix\cite{netflix}, пользователи которого могли выбрать и взять на прокат фильмы на DVD, со временем появилась также возможность смотреть фильмы онлайн. Также на сайте можно выставлять оценки фильмам и оставлять отзывы. В 2006 году Netflix объявили конкурс Netflix Prize\cite{netflix_prize}, в рамках которого предложили всем желающим использовать их датасет с оценками пользователей, чтобы улучшить качество рекомендаций на 10\% по сравнению с использовавшимся на сайте на тот момент алгоритмом. Этот конкурс послужил большим толчком в развитии области рекомендательных систем.


\section{Платформа stepic.org}
\indent Платформа \textit{stepic.org}\cite{stepic} --- русскоязычный сайт, на котором размещено несколько десятков онлайн-курсов, преимущественно технической тематики. Каждый из них прошли от нескольких сотен до нескольких тысяч человек. Ежедневно на платформу заходят тысячи учащихся.
\subsection{Терминология}
\indent Курсы на stepic.org разделены на \textit{модули}, каждый модуль обычно рассчитан на неделю или две. У модуля могут быть установлены дедлайны на решение задач. После жесткого дедлайна пользователь все еще может проходить модуль, но уже не получит баллов за решение задач. Еще у модуля может быть мягкий дедлайн, после которого пользователь получит за решение задач часть баллов.
\\\indent Модуль состоит из \textit{уроков}, каждый из которых посвящен отдельному вопросу. Обычно уроки формируются так, чтобы можно было рассматривать урок в отрыве от курса, то есть как отдельную структурную единицу.
\\\indent В каждом уроке есть несколько \textit{шагов} (\textit{стэпов}), представляющих собой либо лекционный материал в виде видеозаписи или текста, либо задачу одного из множества видов. Бывают задачи-тесты с выбором ответа, задачи с текстовым ответом, задачи на программирование и многие другие. Большинство из них предполагают моментальную автоматическую проверку системой, но есть также задачи на написание эссе, которые студенты сами проверяют друг у друга на основе сформулированных преподавателем критериев.
% вода: типы задач, возможности проведения хакатонов или еще чего
\\\indent Каждый урок или курс может быть помечен \textit{тегом} --- концепцией, которая описывает его содержимое. Примеры таких тегов: программирование, linux, статистика. Каждый тег привязан к объекту в базе знаний Wikidata.
\\\indent Уроки могут быть объединены в \textit{пути}. В отличие от курсов, у путей нет модулей, дедлайнов и возможности получить сертификат за их прохождение. Это просто способ объединять уроки в логическую последовательность. Обычно после того, как курс закрывается для просмотра, его уроки собираются в путь и выкладываются в библиотеку.

\subsection{Возможности}
\indent В основном студенты приходят на stepic.org, чтобы изучать курсы. Успешно прошедшие курс и набравшие нужное количество баллов получают электронные сертификаты. Преподаватели могут установить границу для получения обычного сертификата, и для получения сертификата с отличием.
\\\indent Помимо курсов на stepic.org также есть раздел библиотека, в котором уроки расположены без связи друг с другом. Уроки, которые были включены в один курс, здесь располагаются без связи друг с другом. Сейчас она представляет собой просто список уроков, из которых можно отфильтровать уроки по тегу, либо рассмотреть уроки из одного пути.
\\\indent Stepic.org позволяет пользователям создавать свои уроки и курсы, давать ссылки на них или же встраивать в другие сайты. Также имеется возможность интеграции с любой платформой через LTI, благодаря чему в некоторых курсах на платформе Coursera\cite{coursera} используются задачи, которые создаются и проверяются на stepic.org. Весь контент на платформе распространяется под открытой лицензией Creative Commons\cite{creativecommons}.
\\\indent Пользователи могут общаться между собой с помощью механизмов комментариев под каждым стэпом. Там обычно обсуждают лекционный материал, помогают в решении задач или общаются с преподавателями.


\section{Рекомендательные системы}
\\\indent Согласно общепринятой классификации\cite{rec_sys_handbook}, существует два основных типа рекомендательных систем:
\begin{itemize}
    \item Основанные на \textit{фильтрации контента}. Такие рекомендательные системы рассматривают наборы характеристик пользователей и объектов (фильмов, книг, товаров и т.д.), а затем по этим характеристикам подбирают контент, наиболее подходящий пользователю. В качестве примера можно рассмотреть рекомендацию музыки тех жанров, что нравятся пользователю.
    \item Использующие \textit{коллаборативную фильтрацию}. Для работы этих систем требуется большое количество оценок контента от разных пользователей. Для рекомендации пользователю нового контента система находит пользователей, похожих по интересам на него, и советует то, что смотрели они.
\end{itemize}
\\\indent В основном используются гибридные системы, совмещающие эти два подхода, со специфическими для предметной области деталями. 

\section{Оценка результата}
Для оценки качества рекомендательной системы существует множество метрик\cite{rec_sys_handbook:evaluation}:
\begin{itemize}
    \item \textit{Точность предсказаний.} Метрики такого типа удобны в случае, если рекомендательная система должна предсказать оценку пользователя объекту, например, фильму. Тогда можно оценить отклонение предсказанной оценки от той, что пользователь поставил после просмотра.
    \item \textit{Предсказание потребности.} Для систем, в которых нет явного сбора оценок пользователей, более актуальными становятся замеры заинтересованности пользователя рекомендациями. Примеры таких метрик: точность (доля рекомендаций, которыми пользователь воспользовался, от всех показанных ему), полнота (доля объектов, рекомендованных пользователю, от всех, что могли быть ему интересны) и основанные на них метрики.
    \item \textit{Адаптивность.} Рекомендательная система должна хорошо подстраиваться под интересы пользователя, которые могут меняться со временем. В частности, этот аспект важен для образовательного контента, так как рекомендательная система должна учитывать уровень знаний пользователя и предлагать ему соответствующий контент для обучения.
    \item Менее формализуемые метрики, такие как новизна, доверие пользователя системе, разнообразность рекомендаций и прочие.
\end{itemize}
\\\indent Эти метрики помогут оценить результат работы, а также сравнить разные вариации рекомендательной системы между собой.


\section{Постановка задачи}
\\\indent Задача этой работы --- создать рекомендательную систему, которая могла бы посоветовать студенту на платформе Stepic.org контент, который будет интересен ему, и которая также будет учитывать уровень подготовки студента, его знания и пробелы. Кроме этого система должна уметь оценивать качество контента и его сложность. Единицей рекомендации будет урок.
\\\indent В моей работе будет использоваться гибридная рекомендательная система, объединяющая в себе разные способы рекомендации контента: как фильтрацию контента по различным характеристикам (язык, теги, сложность), так и коллаборативную фильтрацию. Также система будет обладать свойством адаптивности, то есть сможет подстраиваться под меняющиеся знания и интересы пользователя.
\\\indent Для оценки результата и прогресса рекомендательной системы со временем будут использоваться метрики, описанные в предыдущей главе.




\setmonofont[Mapping=tex-text]{CMU Typewriter Text}
\bibliographystyle{ugost2008ls}
\bibliography{diploma.bib}
\end{document}
