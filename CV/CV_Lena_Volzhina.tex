%%%%%%%%%%%%%%%%%%%%%%%%%%%%%%%%%%%%%%%%%
% Medium Length Professional CV
% LaTeX Template
% Version 2.0 (8/5/13)
%
% This template has been downloaded from:
% http://www.LaTeXTemplates.com
%
% Original author:
% Trey Hunner (http://www.treyhunner.com/)
%
% Important note:
% This template requires the resume.cls file to be in the same directory as the
% .tex file. The resume.cls file provides the resume style used for structuring the
% document.
%
%%%%%%%%%%%%%%%%%%%%%%%%%%%%%%%%%%%%%%%%%

%----------------------------------------------------------------------------------------
%	PACKAGES AND OTHER DOCUMENT CONFIGURATIONS
%----------------------------------------------------------------------------------------

\documentclass{resume} % Use the custom resume.cls style

\usepackage[left=0.75in,top=0.6in,right=0.75in,bottom=0.6in]{geometry} % Document margins
\usepackage{hyperref}

\name{Lena Volzhina} % Your name
\address{Saint Petersburg, Russia} % Your address
\address{(921)~77~22~515 \\ LenaVolzhina@ya.ru} % Your phone number and email
\address{ \href{https://github.com/LenaVolzhina}{github.com/LenaVolzhina}}

\begin{document}

%----------------------------------------------------------------------------------------
%	EDUCATION SECTION
%----------------------------------------------------------------------------------------

\begin{rSection}{Education}


{\bf Saint Petersburg State University} \hfill {{\em 2016 --- 2018} \\ 
Department of Analytical Information Systems \\
Master's degree with distinction in Computer Science \& Engineering
%GPA: 4.58
}

{\bf Computer Science Center} \hfill {{\em 2014 --- 2017} \\ 
Contemporary issues in Computer Science, as well as tools for software engineering \\
Modules included (full list here:  \href{https://compscicenter.ru/users/734/}{[ru]}): \\
$\cdot$~Algorithms~and~Data~Structures~\hspace{1cm} 
$\cdot$~Big~Data~\hspace{1cm} 
$\cdot$~Mathematical~Statistics~\hspace{1cm}\\
$\cdot$~Parallel~Programming~\hspace{1cm}
$\cdot$~C++~\hspace{1cm} 
$\cdot$~Java~\hspace{1cm} 
$\cdot$~Python~\hspace{1cm}
}

{\bf Saint Petersburg State University} \hfill {{\em 2012 --- 2016} \\ 
Department of Analytical Information Systems \\
Bachelor's degree in Computer Science \& Engineering 
%GPA: 4.58
}


%{\bf Gymnasium Classicum Petropolitanum} \hfill {{\em 2005 --- 2012} \\ }

\end{rSection}

%----------------------------------------------------------------------------------------
%	WORK EXPERIENCE SECTION
%----------------------------------------------------------------------------------------

\begin{rSection}{Experience}

%\begin{rSubsection}{}{2012 - present time}{Math Tutor}{}
%\item Since school graduating I'm teaching general and olympiad maths, which improves my communication scills and also helps me to explain complex subjects more clearly.
%\end{rSubsection}

%\begin{rSubsection}{Lanit-Tercom, Inc.}{\em Sep 2013 - May 2014}{student project}{}
%\item During the academic year I participated in a student project, aimed at running a quadrocopter. Our task was to keep it in the air and process data from its sensors. 
%\item  At first I was involved in transfering videos from the camera to the client browser, and later I worked on the calibration of the sensors. During this project I improved my teamwork skills and got some practical experience working on a real case.
%\end{rSubsection}

\begin{rSubsection}{Stepik}{June 2015 - September 2017}{Python Developer}{}
\item At Stepik I was involved in different data analysis tasks, mostly related to \textit{content recommendations}. During the first months of work I designed and implemented a simple \textit{rule-based} recommender system for educational content. Later I worked on an \textit{adaptive} recommender system, that uses knowledge graph and psychometric models for the most efficient learning. Also during my time at Stepik I completed many separate backend tasks.
\end{rSubsection}

\begin{rSubsection}{Yandex.Weather}{September 2017 - present time}{Python Developer}{}
\item At Yandex.Weather I'm using \textit{machine learning} methods to make weather forecasts. There are large datasets from forecast providers for analysis and processing, so I need to use various \textit{big data} technics such as MapReduce paradigm. The purpose of my work is to improve the quality of daily weather forecast, calculated by technology named Meteum. I also support and develop code that regularly calculates weather forecasts around the world.

\end{rSubsection}


%------------------------------------------------

\end{rSection}


%----------------------------------------------------------------------------------------
%	TECHNICAL STRENGTHS SECTION
%----------------------------------------------------------------------------------------

\begin{rSection}{Technical Strengths}

\begin{tabular}{ @{} >{\bfseries}l @{\hspace{6ex}} l }
Computer Languages & Python (advanced, Django), C++ (medium), Java (beginner) \\
Tools & Git, SQL, MapReduce \\

\end{tabular}

\end{rSection}


\begin{rSection}{Languages}

\begin{tabular}{ @{} >{\bfseries}l @{\hspace{6ex}} l }
English & Upper Intermediate  \\
Russian & Native Speaker
\end{tabular}

\end{rSection}


% выступления
% * PyLadies 7.08.18 https://youtu.be/DfXsnSDouEo?t=859



%----------------------------------------------------------------------------------------
%	EXAMPLE SECTION
%----------------------------------------------------------------------------------------

%\begin{rSection}{Section Name}

%Section content\ldots

%\end{rSection}

%----------------------------------------------------------------------------------------

\end{document}
